\chapter*{ВВЕДЕНИЕ}
\addcontentsline{toc}{chapter}{ВВЕДЕНИЕ}

Создание музыки творческий процесс, его автоматизация сложна из-за важной роли композитора и труднопонимаемой эмоциональности в музыке~\cite{big}.
Для автоматической генерации музыкальных композиций с учетом эмоционального состояния пользователя-композитора можно использовать трансляционные модели~\cite{web}.

Трансляционные модели --- это способ генерации музыкальных произведений на основе немузыкальных данных, таких как графические образы или текст.
%todo
Данный процесс может быть случайным или основываться на определенных правилах мелодической структуры, с использованием нейросетевых технологий для преобразования и распознавания исходных данных~\cite{translation}.

Генерация музыки на основе изображения может применяется в компьютерных играх, рекламе и фильмах для создания фоновой музыки. 
Автоматизация этого процесса позволит сократить расходы компаний, учитывая небольшие требования к фоновой музыке в этих областях~\cite{actuality}.

%todo
Целью данной работы является изучение (описание) метода генерации фрагмента музыкального произведения по изображению.

Формальная постановка задачи: 
\begin{enumerate}
	\item пусть $A$ --- множество всех возможных изображений, исходные данные для алгоритма;
	\item пусть $B$ --- множество всех возможных музыкальных композиций, результаты работы алгоритма;
    \item Тогда $f: A \rightarrow 2^B$ --- многозначное отображение, которое каждому элементу (изображению) $a \in A$ сопоставляет множество музыкальных композиций $f(a) \subseteq B$.
\end{enumerate}

Задачи этой работы:
\begin{enumerate}
	\item 1;
	\item 2;
	\item 3.
\end{enumerate}
