\begin{definitions}
	\definition{Музыкальный тон}{это устойчивый периодический звук. Музыкальный тон характеризуется его длительностью, высотой, интенсивностью (или громкостью) и тембром (или качество)~\cite{umk}.}
	
	\definition{Темп (итал. tempo)}{это скорость движения в музыке, мера времени в музыке~\cite{umk}.}
	
	\definition{HSB (англ. Hue, Saturation, Brightness --- тон, насыщенность, яркость)}{цветовая модель, в которой координатами цвета являются: цветовой тон (например, красный, зелёный или сине-голубой), насыщенность (варьируется в пределах 0—100), яркость (варьируется в пределах 0—100)~\cite{rgb}.}
	
	\definition{RGB (англ. Red, Green, Blue --- красный, зелёный, синий)}{цветовая модель, описывающая способ кодирования цвета для цветовоспроизведения с помощью трёх цветов, которые принято называть основными~\cite{rgb}}
	
	\definition{Кластеризация}{это разбиение элементов некоторого множества на группы по принципу схожести. Эти группы принято называть кластерами.~\cite{defcluctering}}
	
	
\end{definitions}