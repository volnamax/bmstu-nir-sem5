\chapter{Аналитический раздел}

%todo
В данном разделе описан метод генерации фрагмента музыкального произведения по изображению, алгоритм анализа изображения, 

Генерация фрагмента музыкального произведения по изображению, представляет собой преобразование визуальных данных в последовательности нот с определенным тоном и темпом~\cite{alg}. 

Метод генерации фрагмента музыкального произведения по изображению состоит из двух составляющих алгоритмов: 
\begin{enumerate}
	\item алгоритм анализа изображения; % спросить про алгоритм
%todo
	\item алгоритм генерации музыкального фрагмента с использованием нейронных сетей.
\end{enumerate}	

\section{Алгоритм анализа изображения}

Алгоритм анализа изображения позволяет получить музыкальные характеристики изображения --- определить характер получаемой музыкальной композиции~\cite{alg}.

\begin{enumerate}
	\item Преобразовать входное изображение из цветовой модели RGB в HSB.
	Цветовая модель HSB более удобна, так как содержит необходимые характеристики~\cite{alg};
	\item определить преобладающий цвет изображения;
	\item определить название и цветовую группу преимущественного цвета;
	\item согласно выбранной схеме соотнесения цветов и нот, а также результатах, полученных на предыдущих шагах, определяем тональность произведения.
\end{enumerate}

Тональность и темп --- являются ключевыми параметрами для трансляции изображения в музыку, поскольку они формируют эмоциональную составляющую произведения, и должны быть определены путем анализа цветовой гаммы изображения~\cite{actuality}. Для этого нужно установить соответствие между цветовыми и музыкальными характеристиками \cite{web} (таблица \ref{tab:color_music}). Затем следует определить схему соотнесения цвета и ноты\cite{web}. В статье \cite{colortonote} описывается множество подобных схем, например соотнесение цветов и нот по И. Ньютону, он искал связь между солнечным спектром и музыкальной октавой, сопоставляя длины разноцветных участков спектра и частоту колебаний звуков гаммы, таблица \ref{tab:Newton}.
	
	\begin{table}[ht]
		\centering
		\caption{Соотношение цветовых и музыкальных характеристик}
		\label{tab:color_music}
		\small 
		\begin{tabular}{|c|c|}
			\hline
			\textbf{Цветовые характеристики} & \textbf{Музыкальные характеристики} \\
			\hline
			Оттенок (красный, синий, жёлтый…) & Нота (до, до-диез, ре, ре-диез ...) \\
			\hline
			Цветовая группа (тёплый/холодный) & Музыкальный лад (мажор/минор) \\
			\hline
			Яркость & Октава ноты \\
			\hline
			Насыщенность & Длительность ноты \\
			\hline
		\end{tabular}
		
		\vspace{1em} % добавляем вертикальный отступ
		
		\caption{Соотнесение цветов и нот по И. Ньютону}
		\label{tab:Newton}	
		\begin{tabular}{lc}
			\toprule
			\textbf{Цвет} & \textbf{Нота} \\
			\midrule
			Красный & До \\
			Фиолетовый & Ре \\
			Синий & Ми \\
			Голубой & Фа \\
			Зеленый & Соль \\
			Желтый & Ля \\
			Оранжевый & Си \\
			\bottomrule
		\end{tabular}
	\end{table}

	\subsection {Алгоритм определения тональности}	
		Определение тональности основывается на проведении анализа изображения и использовании таблицы \ref{tab:color_music}, состоит из следующих шагов, описанных в \cite{web}: 
		\begin{enumerate}
			\item преобразовать входное изображение из цветового пространства
			RGB в HSV. Данный шаг позволяет преобразовать изображение к более
			удобному виду, поскольку HSV пространство уже содержит необходимые
			характеристики – название цвета (определяется по параметру hue),
			насыщенность (параметр saturation) и яркость (параметр brightness);
			\item определить преобладающий цвет изображения;
			\item определить название и цветовую группу преимущественного цвета;
			\item согласно выбранной схеме соотнесения цветов и нот, а также результатах, полученных на предыдущих шагах, определяем тональность произведения.
		\end{enumerate}