\chapter{Аналитический раздел}

Создание музыки творческий процесс, его автоматизация сложна из-за важной роли композитора и труднопонимаемой эмоциональности в музыке \cite{big}. Для автоматической генерации музыкальных композиций с учетом эмоционального состояния пользователя-композитора можно использовать трансляционные модели\cite{web}.

\section{Трансляция изображений в звуки}
	Генерация звуков из изображения представляет собой преобразование визуальных данных в последовательности нот с определенным тоном и темпом \cite{alg}. Тональность и темп --- являются ключевыми параметрами для трансляции изображения в звуки, поскольку они формируют эмоциональную составляющую произведения, и должны быть определены путем анализа цветовой гаммы изображения. Для этого нужно установить соответствие между цветовыми и музыкальными характеристиками \cite{web} (таблица \ref{tab:color_music}). Затем следует определить схему соотнесения цвета и ноты\cite{web}. В статье \cite{colortonote} описывается множество подобных схем, например соотнесение цветов и нот по И. Ньютону, он искал связь между
	солнечным спектром и музыкальной октавой, сопоставляя длины разноцветных участков спектра и частоту колебаний звуков гаммы, таблица \ref{tab:Newton}.
	
	
	\begin{table}[ht]
		\centering
		\caption{Соотношение цветовых и музыкальных характеристик}
		\label{tab:color_music}
		\small 
		\begin{tabular}{|c|c|}
			\hline
			\textbf{Цветовые характеристики} & \textbf{Музыкальные характеристики} \\
			\hline
			Оттенок (красный, синий, жёлтый…) & Нота (до, до-диез, ре, ре-диез ... \\
			\hline
			Цветовая группа (тёплый/холодный) & Музыкальный лад (мажор/минор) \\
			\hline
			Яркость & Октава ноты \\
			\hline
			Насыщенность & Длительность ноты \\
			\hline
		\end{tabular}
		
		\vspace{1em} % добавляем вертикальный отступ
		
		\caption{Соотнесение цветов и нот по И. Ньютону}
		\label{tab:Newton}	
		\begin{tabular}{lc}
			\toprule
			\textbf{Цвет} & \textbf{Нота} \\
			\midrule
			Красный & До \\
			Фиолетовый & Ре \\
			Синий & Ми \\
			Голубой & Фа \\
			Зеленый & Соль \\
			Желтый & Ля \\
			Оранжевый & Си \\
			\bottomrule
		\end{tabular}
	\end{table}


		
	\subsection {Алгоритм анализа изображений}
		Алгоритм анализа изображений состоит из 4 шагов:
		\begin{enumerate}
			\item необходимо преобразовать исходное изображение в цветовое пространство HSV. Это преобразование позволяет легко получить общую характеристику каждого пикселя из изображения — оттенок, насыщенность и яркость [3]
		
			\item Второй элемент
			\item Третий элемент
		\end{enumerate}